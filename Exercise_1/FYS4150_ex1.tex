\documentclass[11pt,twoside,a4paper]{article}


\usepackage[T1]{fontenc}
\usepackage[english,norsk]{babel}
\usepackage[utf8]{inputenc}
\usepackage{xcolor,graphicx,wrapfig,lipsum}
\usepackage{amsmath}

\usepackage{array} 
\usepackage{multicol}
\setlength{\columnsep}{2cm}
\usepackage{multirow}
\usepackage{pdfpages}
\usepackage{float}
\usepackage{standalone}
\usepackage{tikz}

\usepackage{caption}
\usepackage{subcaption}
%\usepackage[export]{adjustbox}

\usepackage[margin=2cm]{geometry}

\newcommand{\nl}{

\medskip
\noindent

}


\pagestyle{plain}


\begin{document}


\pagestyle{plain}

\begin{titlepage}
	\vspace*{3cm}
	\centering
	{\scshape\LARGE Universitetet i Oslo\par}
	\vspace{1cm}
	{\scshape\Large Computational physics\par}
	\vspace{0.5cm}
	{\scshape\Large FYS4150\par}
	\vspace{1.5cm}
	{\huge\bfseries Exercise 1\par}
	\vspace{2cm}
	{\Large\itshape Heine H. Ness \par}

	\vfill

	{\large \today\par}
\end{titlepage}


\section*{Introduction}

In this assignment we are going to solve a linear second-order differential equation numerically.
\nl
The programming language of choice is c++ and all programming was done in Qt.

\section*{Metod}
\subsection*{Differential equations}

Alot of physics problems involve solving a linear differential equations. Examples of these are the Schrödinger equation, diffusion equation and poisson's equation.
\nl
The thing all these equations have the form 

\begin{align*}
\nabla f(x) = cf(x)
\end{align*}

where $\nabla = \frac{\partial^2}{\partial x^2}$ the second order partial derivative. $c$ is a constant and $f(x)$ is a known function.

\subsection*{Nummerical derivation}

Derivation decribes the curvature of a function and the second order derivative decribes its slope. The general formula for the first order derivative is

\begin{align*}
\frac{\partial}{\partial x} = \frac{x_i - x_{i-1}}{h}
\end{align*}

\end{document}